\documentclass[a4paper, twoside]{article}

\usepackage{verbatim}

\setlength{\oddsidemargin}{0in} \setlength{\evensidemargin}{0in}
\setlength{\textwidth}{6.2in}
\setlength{\topmargin}{-0.3in} \setlength{\textheight}{9.8in}

\title{CS23710 - Runners and Riders}
\author{Tom Leaman (thl5)}

\begin{document}
\maketitle

\section{Design}
\subsection{Program loop \& main}
The main function (in main.c) is a very simple function which starts by reading
in the data from the user supplied filenames and then executes a while loop
which displays the menu, prompts for input and then calls helper functions
based on what the user requires from the program.

\subsection{Vector}
I decided to implement an array-backed vector to store the data being passed to
the program. As the code does not know the size of the file before reading, I
felt that a dynamically resizable data structure would be preferable and a
vector seemed like the simplest option. I implemented all of the functionality
that I felt would be necessary (including originally a remove function which
proved to not be needed) and tested it with vectortest.c.

The vector uses a doubling strategy when it becomes full which means that it
will only resize every $2^n$ inserts. I think the benefit of readable,
understandable (I hope) code out-weighs the extra memory allocated at the
end of the vector.

I did contemplate implementing a generic linked list structure as well but I
actually don't believe it to be necessary; the only time data is being shuffled
around is when the vector resizes or when it is being sorted. The sorting
algorithm used is bubble sort which means that during any one operation, only
two sections of memory are being swapped. This (as far as I am concerned)
eliminates the benefit of using a linked list (the linked list would also have
more memory overhead per item due to each item needing a pointer to the next
node).

\section{Compilation}

\section{Output}

\section{Source}
\subsection{vector.h}
\verbatiminput{vector.h}
\subsection{vector.c}
\verbatiminput{vector.c}
\subsection{vectortest.c}
\verbatiminput{vectortest.c}
\subsection{util.h}
\verbatiminput{util.h}
\subsection{util.c}
\verbatiminput{util.c}
\subsection{data.h}
\verbatiminput{data.h}
\subsection{node.c}
\verbatiminput{node.c}
\subsection{track.c}
\verbatiminput{track.c}
\subsection{course.c}
\verbatiminput{course.c}
\subsection{entrant.c}
\verbatiminput{entrant.c}
\subsection{event.c}
\verbatiminput{event.c}
\subsection{main.c}
\verbatiminput{main.c}

\section{References}

\end{document}
